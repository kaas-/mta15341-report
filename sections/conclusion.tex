%!TEX root = ../master.tex
\chapter{Conclusion}\label{ch:conclusion}
Based on the results from the testing and evaluation of the prototype for the augmented Terra Mystica board, both usability- and technical-wise (Chapter \ref{ch:testeval}), an answer can be given to the problem statement from Section \ref{sec:ProblemStatement}.

\section{Problem Formulation}
\textit{“Can Terra Mystica become more usable through the use of a computer vision augmented board…”} According to the results from this project, the answer is ‘not at the prototype’s current state, but possibly in the future’. When testing the prototype, it became clear from the participants usage that it slowed down the pace of the game because functions were not working efficiently. It was the terraforming and turn mechanics that took longer time than it did in the original version of the game, which was caused by the users having trouble doing those actions. Despite this, some of the participants said they expected the actions not to be an hindrance at all if their functionality worked flawlessly. Some participants expected a faster pace of the game, if the augmented version had functions which took care of a player’s resource management. Furthermore, participants saw a potential in the augmentation interface design, since they believed its simplistic design would benefit new players and that it could have a more explanatory GUI, which for the new players might increase the game’s usability. All this will require an improved version of the prototype. It then should be tested in order to see if those ideas do increase the game’s usability. This is why at its current state it might only in another version make Terra Mystica become more usable through the use of a computer vision augmented board.

\textit{“...while still maintaining the feeling of playing a traditional board game?”} With the results in mind, the answer here is ‘the prototype kept the players feeling like they still were playing a traditional board game, although the augmentation could change the game’s purpose’. A consensus between the participants were how they all still felt like they were playing the same game in both its versions, despite the difference in how some actions in the game were executed. But there were still participants who saw another change in the game’s purpose from the original to the augmented one. Some saw the augmented version as a more simpler, larger and more clear version of the two. Others saw the augmented board game as a more dramatic game to play, since it was a game where you instead of placing tokens for terraforming used your hands to do it. Additionally some saw the two versions of the games as two ways of playing it, since one was at a large box you had to stand up to use, while light was shining from the board, and the other was a smaller more tactile version were you sat down and closer to each other. In conclusion the participants still felt like they were playing Terra Mystica and it maintained the board game feeling even if the different versions could represent different purposes of use.

\section{Success Criteria}
\textit{"The project must have gone through evaluations and iterations, based on relevant usability principles."}

The prototype was developed through multiple stages, first being tested as a paper mock-up. After this initial lo-fi testing, the feedback led to the second stage of development, which made use of the RDI setup for the use of terraforming. This stage was then tested with more participants, whose feedback led to establishing step for further implementations of the augmented board game.

\textit{"Augmentation technology must be utilised with Terra Mystica in such a way that it does not take away from the experience of the game but enhances it."}

In the usability testing of the prototype, many participants expressed positive interest in the board, when asked how they would respond if the prototype was developed into a fully implemented product. However, when asked if the prototype could replace the board game in its current state, the participants answered much more negatively, as technical issues and difficulties with RDI slowed the game’s pace and made hand gesture detection unreliable. Although some players reported that the simpler graphics of the augmented board was easier to get an overview from, others said the same from the more detailed pictured tiles of the original board. Therefore, is this success criterion not fulfilled, with the prototype in its current state.

\textit{"Implemented augmentation technologies should utilise computer vision to augment Terra Mystica"}

By making use of an RDI setup in order to detect gestures in the augmentation, the prototype removed the use of cardboard disks when terraforming. Instead it makes use of the player placing their hand on the table, in order to terraform and change turns. Therefore, this success criterion has been achieved.

\textit{"The accuracy and effectiveness of the image detection should be tested in a proper framework."}

The image detection implemented in the prototype was tested according to the framework described in Subsection \ref{subsec:techtest}, where accuracy, effectiveness, and overall efficiency of the system is tested. Appendix C documents the results of the testing carried out. However, if it is a proper framework is debatable, as it is not built upon previous known methods of technical testing. Nonetheless, because of its defined procedural structure, and clear display of data, this success criterion is considered to be achieved.