%!TEX root = ../master.tex
\chapter{Final Product}\label{ch:finproduct}
The following chapter contains a description of the state of the product, at the end of the project period.

\section{Physical product}
The table itself is constructed by aluminium extrusions, measuring 120 x 102 x 89 cm. A shelf is located on the internal side of the top frame, which is used to hold a 3 mm acrylic glass plate. The underside of the glass is covered in a single layer of baking paper. This acts as the diffusion material in the RDI setup, as well as the screen for the short-throw projector.

In the bottom of the table are extrusions, which hold the rest of the RDI setup, and the projector. The camera is placed in the bottom of the table, so that the area of sight is larger. The projector is placed to one side of the table, as this makes the cone of projection land central on the screen. The infra-red lights are placed 30 cm into the air, by mounting them on top of extrusions. This height is optimal, as the camera can detect more, the closer the lamps are to the screen. However, the closer the lamps are to the screen, the smaller are the cones of infra-red light.  Furthermore, if the lamps were placed higher, the extrusions they were mounted on would start blocking the projector, making parts of the screen not viewable by the users. To contain the infra-red light inside the table, a black cloth cover is placed around the sides of the table.

A computer runs the software, and is connected to the camera and the projector. The camera acts as input for the software, and the projector shows the output on the screen.

\section{Digital product}
The product displays a recreation of the Terra Mystica board and include the following:
\begin{itemize}
	\item The hexagonal tiles in the seven different colours, plus the river tiles
	\item An inner frame around the board tiles, which is used to track each players' amount of victory points.
	\item An outer frame, which contain a table for how to score victory points at the end of each round, and at the end of the game. It also contains an undo button, and button for actions that can be taken by only one player each round.
	\ldots
\end{itemize}



\todo{remember player select area was scrapped because of bad illumination at the edges.}
\todo{remember the undo bottom was planned, but scrapped because of time}