%!TEX root = ../master.tex
\chapter{Discussion}\label{ch:discussion}

\section{Validity and reliability of test}
dink dink dink

\section{Further Implementation}
Aside from the product's minimum implementation requirements, it is possible to make further iterations, suggested in this chapter. These are based on design ideas for the program from before its implementation. There are also further implementation ideas based on opinions and suggestions from the evaluation results and the experiences after the technical test.

\subsection{Improvement of terraforming}
Although the implementation of an undo button was originally planned, it did not become a part of the current version. Participants of the usability test (described in Chapter~\ref{ch:testeval}) have voiced the need for such a function, as well as provided ideas to how such a function could be implemented (Section~\ref{satisfactionResult}). It could be a button, which is a visible part of the board's GUI, placed away from the game map. This button could revert the last terraforming action taken in the game. An algorithm for this function could be a stacking system that remembers previous actions, allowing the program to revert them, returning to previous states of the program. Another solution could be an expansion of the terraforming action. Anyone, regardless of whose turn it is, should then be able to change a tile to any type of terrain in the game through a multiple choice menu. This would help users undo mistakes by letting them change the tiles back to what the player originally intended or back to their original state. A possible problem with the multiple choice solution is its dependency on players' memory regarding what the terrain of the hexes originally were before the misplaced terraforming. 

\subsection{Improvement of taking turns}
\todo{Daniel demands an addition of some technical further implementation in this subsection}Participants of the usability test in Section~\ref{ch:testeval} had ideas how to improve the turn taking further, besides improve in its programming. They mentioned that instead of having the turn switching be by having a hand hovering/pressing against the middle of the surface, one should should be able to switch turns by interacting with an area specifically only for switching turns. This solution could  also make its programming somewhat easier, since it could base its gesture recognition on location, rather than it being based on shape\todo{"...which was a problem in the technical test" - or was it? Someone confirm+add this or remove this suggestion}.

\subsection{Turn specification}
\todo{How many times do I need to reference to the test and evaluation? Maybe I should just reference it in an introduction to all the further implementation suggestions?}Besides improvement of the turn taking in the augmented game game, test participants also noted they would like to have a specification of whose turn it would be and an overview how the turn order is at any at any given time. The current player's turn could for example be specified in the game's GUI by having the player's personal area of the board lighten more up than the other players'. Another example could be by giving each player a color, maybe just their terrain color, which would light up space not used on the board, in order to emphasise which player currently is in charge of the turn.
Regarding the turn overview, a solution to that could be an overview implemented as part of the board's GUI. Examples of the overview's placements could either be as a part of the board's game global information, just like the game objectives, where it would have to be large enough and well placed enough for all the players to evenly observe. It could also instead be placed in front of every player, giving them each a equal personal overview of the turn order, and in addition one could design the overview so each overview graphic had focus on the player's placement in the turn order. The latter version of the turn overview's advantage could be an easier overview for all players, since none would have to look at the turn order from an odd angle, but the first version advantage is that it might in the end take up less space than having an overview for each player, since the overviews still needs to be comprehensible.

\subsection{Digital race/faction card}



%Another additional function to include could be for each player to choose their own faction from the game. Also, one should be able to add a colour to that faction, so it would display the chosen colour on the board when it was the faction’s turn. When a faction is selected, its counterpart (the backside of a faction’s card), should be inaccessible.

%Lastly, the global power actions mentioned in Section \ref{sec:BoardInterface} could be an interactive function instead of spaces that you put game pieces on. The idea is to make the players able to press the powers, which would then make the powers become inaccessible. An advanced version of this function could have the program register faction rules, which would make the global actions still accessible to them. 

%\subsection{Player assistance}
%In order to assist players and streamline the game pace, the following further implementations could be done.

%The program could remember which players have placed their buildings on each tile. When a player places or upgrades a building next to a tile which is occupied by another player, the game could remind one player to offer the other power. Ideally, even the specific type of building already occupying the surrounding tiles is remembered by the program, allowing it to calculate exactly how much power is to be offered to a player.

%As of now, the interface has the global goals for each round displayed, but they are fixed, as opposed to randomised like in the original game. A possible iteration for this is to make the program draw randomised one-round goals in the beginning of the game, keeping each game unique. Goals for previous rounds that are no longer relevant could be hidden.

%Another possible iteration is to make the colour of the currently selected player visible on the board, perhaps in the form of a border around the board. This way, players will not be in doubt whether or not they have correctly selected a player.

%To assist the player with the management of their points during and at the end of the game, a feature which does the point registration and mathematical tasks could be added. In order for this to work, some subtraction and addition in a player’s amount of points should happen due what happens in the game and what the state of the game is at its end. Examples of such scenarios are: when a player receives power that would make them lose points, when a player reaches specific goals in the game, when a player places a specific building, etc. 

%In the original game, each player has a card containing all information about their chosen race. This card could be made digital. This would open up the opportunity for players to ask the game for tips about specific icons and what they mean, in case they fail to remember.

%Furthermore, the card keeping track of how many points each player has with each cult could be made digital. This digital board could react when players pass the given milestones on the track, reminding them to receive power. If this and the aspect overseeing goals are both implemented, the cult track could even award the players with the correct one-round bonuses currently in effect.  

%\subsection{Technical tweaks}
%As the original game can be played by 2-5 players, an ideal iteration of the program is to widen its capacity to 5 players. For a more advanced implementation, the board's GUI could change, due to the numbers of players in the game.

%To achieve more precise BLOBs and make the program respond better to user input, a background subtraction step can be added to the segmentation process.