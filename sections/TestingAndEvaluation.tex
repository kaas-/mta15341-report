%!TEX root = ../master.tex
\chapter{Testing and Evaluation}\label{ch:testeval}
writing something here to see if it works

\section{Purpose of test}
The purpose of this test is to test usability and functionality of the product. The ISO 9241 \citep{ISO} standard defines usability as effectiveness, efficiency, and satisfaction of the user. Put more precisely, this means:
\begin{itemize}
\item Efectiveness: To what extent are the goals of the product achieved?
\item Efficiency: What resources (including time) are expended to achieve the goals?
\item Satisfaction: To what extent does the user find the system acceptable?
\end{itemize}
With these in mind, the test aims to clarify whether the product lives up to its test objectives.

\section{Test objectives}
Our problem statement is defined in Section~\ref{sec:ProblemStatement}. Based on this problem statement, it is important not only to test how well the program functions on a technical level, but also if the test participants still feels like they are playing the original board game, which will be tested on a level of satisfaction/acceptance.
The success criteria (Section~\ref{sec:SuccessCriteria}) and the system requirements (Section~\ref{sec:ReqSpec}) each adds to the specifications on the technical and satisfactory \todo{is the project success also a factor of the test objectives??} objectives of the test.
These parts together make up these test objectives:

\begin{enumerate}
\item Do the players still feel like they are playing Terra Mystica, when they are playing the augmented version? \todo{this might need specifications or a broader question}
\item How does the players deem the board game augmentation's usability?
\item How well are players able to take turns via the augmentation? \todo {Is it supposed to be like this? Is it in methods that we first decide how to compare results to objectives?}
\item How well are players able to terraform? \todo{should this be in comparison to for example original terraforming with game pieces?}
\item How well are players able to undo mistakes they did while terraforming?
\item How is is the players' use of the rest of the gameboard, and it is similar to how the original game worked?
\item How hindered is the pace of the game by the three actions each player can take (chose player, terraform and undo)?
\end{enumerate}

These objectives sets the goals and purpose for the evaluation methods. 

\section{Method}
The test methods used, their criteria and their measurement are meant to see if the test objectives are met.

\subsection{Test set-up methods}
The test will take place in a controlled setting, were nobody else than the test participants and testers will be present around the board while a game is taking place. The participants will be playing, and the testers will only act as observers who acts as documenters, and helpers/guides - in case the participants have any technical problems.
The documentation of the test will be done by a tester filming the participants playing the game and a note-taker that also acts as the spokesperson for the testers.
\todo{pick up from here DAN}

\subsection{Measurable test methods}
\begin{enumerate}
\item 
\end{enumerate}

\section{Results}
text\todo{analysis and conclusion}Testing a ref to Larsen \citep{TestingLecture}