\chapter{Initial Problem}\label{ch:iniprob}
In this section, the initial ideas for this project will be explained, as well as the criteria they were based on.

The initial problem formulation is based on the semester theme, visual computing -- human perception, as well our chosen project recommendation, augmented board games. Because of this, we wanted our initial problem formulation to be based on augmenting a board game by use of computer vision. As we tried to specify what our initial problem formulation should cover, we generated several ideas through brainstorming.

For each idea, we noted down which board game to augment, chosen technology needed, and a rough plan for the process. Furthermore, we decided that all ideas had to fulfil certain criteria, which are as follows:
\begin{enumerate}
	\item The idea must implement computer vision, since that is what we will need to fulfil our curriculum.
	\item The idea must include an existing board game that we have access to. We will not design our own board game from scratch.
	\item The implemented computer vision for the idea must be interactive. This was important for us to remember, in order to stick with the main course of 3rd semester in the curriculum.
	\item The chosen board game must not have an average game time of over two hours. This criterion was based on the assumption that if the game's playtime is too long, it will be harder to test it properly.
	\item The implemented computer vision must not make the board game itself more convoluted than it is to begin with. The product should augment the board game, not make it less enjoyable to play, and complicating a game further without adding to it would defeat the purpose of the project.
\end{enumerate}

((Here there should be a list over our project ideas. I an not sure of what level of details are needed here, and how much of it should be in the appendix instead of this chapter - also, maybe there should even be a small detailed description of Terra mystica...not written by me))

((Emilie note: the following should probably go under ideas. Yes, no, maybe?))

We chose to center our project around an interactive board for the game Terra Mystica. For this game, we can project the board onto a table, while physical game pieces on the board are detected using computer vision. The following tasks need to be done in order to make this product a reality:

\begin{enumerate}
\item First, we need to make the board for the projection. For this, the board's functionality will be prioritised over aesthetics, as the most important factor is that the tiles can be detected through computer vision. We would then need to properly project this board.
\item We then need the computer vision to recognise the hexagons that make up a Terra Mystica board.
\item After that, the projected board must be able to change the colours of the hexagons on it, since that is an essential visual part of the game itself.
\item The computer vision now needs to recognise the game pieces.
\item After the hexagons can be recognised and change colour while the game pieces also are being recognised, we will need to make the program able to calculate when a certain hexagon should change colour depending on the game pieces.
\end{enumerate}

\section{Ideas}\label{sec:idea}

\subsection{Terra Mystica Explained}

\subsection{Target Group Explained}

\section{Final Problemformulation}\label{sec:finalprob}