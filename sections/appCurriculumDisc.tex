%!TEX root = ../master.tex
\chapter{Reader's Guide to This Report's Relevance to the Semester Requirements}\label{app:semesterRequirements}
The requirements for the project module of this university semester are listed as follows:

\textbf{Knowledge}
\begin{enumerate}
\item Must have knowledge about the terminology within visual computing.
\item Must be able to understand how a particular visual computing system, e.g., the semester project of the student, works.
\item Must be able to understand and compare a particular visual computing system, e.g., the semester project, to similar systems and to the surrounding society.
\item Must be able to understand and explain the mathematical fundamentals of visual computing.
\newcounter{enumTemp}
\setcounter{enumTemp}{\theenumi}
\end{enumerate}

\textbf{Skills}
\begin{enumerate}
\setcounter{enumi}{\theenumTemp}
\item Must be able to analyse a problem and (if possible) suggest a solution that uses relevant theories and methods from visual computing.
\item Must be able to analyse a system that is based on visual computing and identify relevant constraints and assessment criteria. This relates both to the usability of the system, the technical aspects of the system and (if relevant) the usefulness to society 
\item Must be able to design and implement (apply), a system (or parts hereof) using relevant theories and methods (if possible) from visual computing.
\item Must be able to test and evaluate (analyse) a visual computing system (or parts hereof) with respect to the aforementioned assessment criteria.
\item Must be able to communicate the above knowledge and skills (using proper terminology) both orally and in a written report.
\setcounter{enumTemp}{\theenumi}
\end{enumerate}

\textbf{Competencies}
\begin{enumerate}
\setcounter{enumi}{\theenumTemp}
\item Must be able to discuss relevant theories and methods of visual computing and general theories on perception and apply to concrete problems and situations.
\item Must be able to apply current knowledge on human perception and visual computing in the evaluation of an implemented systems.
\end{enumerate}
\par

Requirement no. 1 is not covered by any specific chapter, as it is demonstrated simply through the correct usage this terminology throughout the project report. However, Chapter~\ref{ch:implementation} contains the most heavy usage of relevant terminology.

Requirement no. 2 are covered by Chapter~\ref{ch:implementation} and Chapter~\ref{ch:codeover} which provide a technical description of the semester project.

Requirement no. 3 are covered in the State of the Art- and Competitive Analysis sections of Chapter~\ref{ch:bgres}.

Requirement no. 4 is covered in the Image Processing section in Chapter~\ref{ch:implementation}.

Requirement no. 5 is covered in part by Chapter~\ref{ch:iniprob}, Chapter~\ref{ch:bgres}, and Chapter~\ref{ch:finprob} which analyse the problem, and in part by Chapter~\ref{ch:design} and Chapter~\ref{ch:implementation} which propose a solution.

Requirement no. 6 is covered by Chapter~\ref{ch:finprob}, as well as Chapter~\ref{ch:testeval}.

Requirement no. 7 is covered by Chapter~\ref{ch:implementation} and Chapter~\ref{ch:codeover}.

Requirement no. 8 is covered by Chapter~\ref{ch:testeval}.

Requirement no. 9 is covered by the report and presentation at the exam.

Requirement no. 10 is covered by Chapter~\ref{ch:bgres} and Chapter~\ref{ch:implementation}.

Requirement no. 11 is covered by Chapter~\ref{ch:testeval} and Chapter~\ref{ch:discussion}.
