%!TEX root = ../master.tex
\chapter{Further Implementations}\label{ch:furthimp}
Hurdy dur intro\todo{we need an intro here}\todo{this whole chapter will be reiterated after evaluation}

\section{Additional functions}
Further implementation is possible in the functions of the augmented board game. Although we had originally planned to implement an undo button, it is not a part of the current version. If that function should have been included, it should have been an interactive area of interest separate from the map. This area should also be a visible part of the board’s GUI. An algorithm for this function could be a stacking system that remembers previous actions, allowing the program to revert them, returning to previous states of the program.

Another additional function to include could be for each player to choose their own faction from the game. Also, one should be able to add a colour to that faction, so it would display the chosen colour on the board when it was the faction’s turn. When a faction is selected, its counterpart (the backside of a faction’s card), should be inaccessible.

Lastly, the global power actions mentioned in Section \ref{sec:BoardInterface} could be an interactive function instead of spaces that you put game pieces on. The idea is to make the players able to press the powers, which would then make the powers become inaccessible. An advanced version of this function could have the program register faction rules, which would make the global actions still accessible to them. 

\section{Player assistance}
In order to assist players and streamline the game pace, the following further implementations could be done.

The program could remember which players have placed their buildings on each tile. When a player places or upgrades a building next to a tile which is occupied by another player, the game could remind one player to offer the other power. Ideally, even the specific type of building already occupying the surrounding tiles is remembered by the program, allowing it to calculate exactly how much power is to be offered to a player.

As of now, the interface has the global goals for each round displayed, but they are fixed, as opposed to randomised like in the original game. A possible iteration for this is to make the program draw randomised one-round goals in the beginning of the game, keeping each game unique. Goals for previous rounds that are no longer relevant could be hidden.

Another possible iteration is to make the colour of the currently selected player visible on the board, perhaps in the form of a border around the board. This way, players will not be in doubt whether or not they have correctly selected a player.

To assist the player with the management of their points during and at the end of the game, a feature which does the point registration and mathematical tasks could be added. In order for this to work, some subtraction and addition in a player’s amount of points should happen due what happens in the game and what the state of the game is at its end. Examples of such scenarios are: when a player receives power that would make them lose points, when a player reaches specific goals in the game, when a player places a specific building, etc. 

In the original game, each player has a card containing all information about their chosen race. This card could be made digital. This would open up the opportunity for players to ask the game for tips about specific icons and what they mean, in case they fail to remember.

Furthermore, the card keeping track of how many points each player has with each cult could be made digital. This digital board could react when players pass the given milestones on the track, reminding them to receive power. If this and the aspect overseeing goals are both implemented, the cult track could even award the players with the correct one-round bonuses currently in effect.  

\section{Technical tweaks}
As the original game can be played by 2-5 players, an ideal iteration of the program is to widen its capacity to 5 players. For a more advanced implementation, the board's GUI could change, due to the numbers of players in the game.

To achieve more precise BLOBs and make the program respond better to user input, a background subtraction step can be added to the segmentation process.