\chapter{Initial Problem}\label{ch:iniprob}
Here we will explain our initial ideas and criteria we based our choices on.

We have based our initial problem formulation on our semester theme and our chosen project recommendation. Because of this, we wanted our initial problem formulation to be based on computer vision and a augmented board game, with the augmentation being based on computer vision. As we tried to specify what our analysis of the initial problem formulation should cover, we generated several ideas through brainstorming.

Those ideas each had a chosen board game, chosen technology needed for the ideas and a rough project process planning. Furthermore, all the ideas went through some criteria, in order to filter our ideas.
The criteria were:
\begin{enumerate}
	\item The idea must implement computer vision, since that is what we will need to fulfil our curriculum.
	\item The idea must include a existing board game that we have access to. So we do not design our own board game from scratch.
	\item The implemented computer vision for the idea must be interactive. This was important for us to remember, in order to stick with the main course of 3rd semester in the curriculum.
	\item The playtime of the idea's chosen board game must not have an average game time over two hours. This criteria was based on the assumption that if the game's playtime was too long, it would be harder to test it properly.
	\item The implemented computer vision may not make the board game itself more convoluted than it is. This criteria was added after a discussion, where the study group agreed on that we would not want to make the board games 'harder to play' than they were. ((maybe elaboration of rewording needed here).
\end{enumerate}

((Here there should be a list over our project ideas. I an not sure of what level of details are needed here, and how much of it should be in the appendix instead of this chapter - also, maybe there should even be a small detailed description of Terra mystica...not written by me))

The chosen project idea were a interactive board for the game Terra Mystica. We believed that such a project would require a projection of the board, while an implementation computer vision should detect physical game pieces on top of the board.
Our understanding of our chosen project idea lead us to make up an assumption of which things we would need to do in order to make our idea a reality:
\begin{enumerate}
\item First, we would need to make up the board for the projection. Here we imagined that the board's function and usability (project-wise)would be prioritised over cosmetics. We would then need to make the board a projection.
\item We then need the implemented computer vision to recognise the hexes that from a projected Terra Mystica board.
\item After that, the projected board must be able to change the colors of the hexes in it, since that is an essential visual part of the game itself.
\item The computer vision now needs to recognise the game pieces.
\item Now that the hexes can be recognised and change color while the game pieces also are being recognised, we will need to write/code(??!?) a program that can tie the hex color change to the game piece placement.
\end{enumerate}

\section{Ideas}\label{sec:idea}
When starting to generate ideas for the project, we made a structured brainstorm. We wrote the topics "Board games" and "Computer Vision" on a blackboard each. Then we wrote whatever came to mind between the individual group members. This gave the following result

\begin{tabular}{l | l}
Board games & Computer Vision\\
\hline
Terra Mystica & React on tokens and gestures\\
Betrayal & Projection - Interactive\\
Classical Games & Recognise:\\
Strategy Games & - Colour\\
Settlers & - Dice\\
Card Games & - Etc\\
Dominion & Maths\\
Pandemic & \\
New Risk & \\
\end{tabular}

Afterwards, we started to look at the two lists, and combine them together under the topic "Fusion". Our brainstorm table then looked like this:

\begin{tabular}{l | l | l}
Board games & fusion & computer vision\\
\hline
Terra mystica & Points & React on tokens and gestures\\
Betrayal at house on the hill &  Virtual board & Projections\\
Classic games & Variation board & Recognise: \\
Strategy games & Dice rolls & - Colour\\
Settlers & System of cards & - Dice\\
Card games & Theme & - Etc\\
Dominion & Virtual setup template & Maths\\
Pandemic \\
New risk\\
\end{tabular}

From this brainstorm, we expanded the ideas by taking specific board games, and discussing how those games can be augmented through computer vision.
\subsection{Terra Mystica Explained}
Terra mystica is a strategy board game, for 2-5 players where the winner is the player with the most victory points. In the beginning of the game you choose 1 of 14 races that you are going to play. Each race have special abilities, some of which let you gain victory points based on certain conditions, where others give you the ability to expand faster. 
When all players have chosen a race the game begins, the first player places a dwelling and the player to the left then places theirs. When the last player places dwellings they place two, then the player to the right places their second. When it's the starting player again the game begins. You then expand upon the dwellings placed and try to obtain victory points. Victory points are obtained throughout the game, based on your race and the different objectives.
There are also certain objectives each round, which give you victory points based of different things, such as each time you place a dwelling this round you gain 2 victory points.  
Two objectives are always the same, achieve the highest amount of cohesive hexagons, and have the highest amount in the four different cult, which can be gained by using priests that you obtain from building temples, which also grant you a favour tile that increase you cult and might give you an additional effect. When the game is over you then gain points based on progress through the 4 cult tracks individually, the player with the most points gain more then the second place. 

\subsection{Target Group Explained}
14+
\section{Final Problemformulation}\label{sec:finalprob}