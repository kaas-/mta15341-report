%!TEX root = ../master.tex
\chapter{Discussion}\label{ch:discussion}

\section{Validity and reliability of tests}

\subsection{Validity and reliability of usability test}
When considering validity and reliability of the tests conducted one need to have some strategies to base it on. The strategies used in this section have been suggested by Creswell \citep{Creswell}.

Regarding validity they can be summed up to these points:
\begin{itemize}
\item Triangulate different sources of information and establishing themes.
\item When describing the findings it needs to be richly detailed, to represent the results as realistically as possible.
\item Clarify that the researcher is biased and self-reflect on the results.
\item Present the negative feedback and the feedback that does not fit with the majority to show that the different perspective have been considered.
\end{itemize}

As seen in section \ref{satisfactionResult} the results have been described thoroughly and present both the positive feedback that back up our problem formulation and the negative feedback that does not back it up. This is consistent whenever a new theme in the results is established, which gives the results validity. Generally throughout the test the guides have been encouraged to be as unbiased as possible. However since it is not possible to put all ones own bias entirely away the questions asked and the results and themes reached are of course made on a basis of our bias of the product we have created. We have however as can be seen throughout the entire Chapter \ref{ch:testeval} considered all angles offered to us concerning the test results. This can be argued as giving our results a sufficient degree of validity. In hindsight, more could have been done to raise the degree of validity, as for example bring in test participants who were not fellow students and therefore would not have the bias gained from the same study. 


Reliability deals with the test set-up and procedures. The points Creswell \citep{Creswell} makes on this topic can be summed up as follows:

\begin{itemize}
\item Check transcripts for mistakes after transcribing.
\item Make sure that the guidelines and procedures remain the same throughout the process.
\item Make sure all group members are aware of the guidelines, procedures and results. 
\item Compare results reached independently.
\end{itemize}

There were not made transcripts of the final tests, since this would take too much time. Therefore we cannot argue for reliability on that point. However we kept to the same guidelines and script for all the tests which all the group members were aware of since different people conducted the tests. All the results were compared in the analysis in section \ref{subsec:analysis}. This all gives a high degree of reliability of our results. 

\subsection{On reliability in technical test and prototype}
This section will discuss the implications of calibration and bugs in the prototype use and testing. 
\subsubsection*{Calibration and reliability}
In order for the prototype to work, a lot of time has to go into setting it up, also called calibration. The IR lamps have to have the right angle and run for a while. The program has to be restarted at times, which also removes noise \todo{for some reason we don't know o.O}. Despite of all this, a lot of problems still remain as can be read below. This makes for poor reliability, as it is hard to test a prototype with such a fickle nature. The chance that you get the same parameters set up for testing twice are challenged and only overcome by several modifications and nudges.

While performing the technical test the people responsible kept a bug log as a tool to track common errors. The bug log can be seen in \ref{app:techTest} \todo{appendix missing} in its entirety.

The most common errors were ordered in tags: Neighbour Change (NBC), Shadow Threshold failure (SWT), and Area Detection Failure (ADF). 
\subsubsection*{Bug - NBC} 
NBC was a common error closely linked SWT, meaning that almost every occurence of NBC had a SWT bug as well. This could indicate that NBC wasn't really to blame, but SWT. Regardless, NBC pertains to the hexagon's centre and the centre of the convex hull (of the fingers trying to change it) are misaligned.
\subsubsection*{Bug - SWT} 
SWT was the most common area in the technical tests but also during the user evaluation tests. This bug has specifically to do with more than the fingers or hand being detected, such as parts of an arm. Furthermore, the bug can occur when a hand is hovering close to the board, but not touching it.
\subsubsection*{Bug - ADF} 
The ADF bug occurred most commonly when a player change was intended but a small-enough hull was made, resulting in a unintentional tile change.
\subsubsection*{Bug remedy}
Certain things could be done to remedy these bugs, such as:
\begin{itemize}
	\item Calibrating the threshold to avoid hovering elements, such as arms or close objects are not added as objects.
	\item Performing intelligent (with a learning rate) background subtraction to cancel out noise and ensure an easier thresholding. This would also reduce the calibration time.
\end{itemize}

\section{Further Implementation}
Aside from the product's minimum implementation requirements, it is possible to make further iterations, suggested in this chapter. These are based on the evaluation results from both the user test and the technical test, which is described in Chapter~\ref{ch:testeval}. The implementation suggestions are based in the results that are considered valid and reliable.

\subsection{Improvement of terraforming}
Although the implementation of an undo button was originally planned, it did not become a part of the current version. Participants of the usability test have voiced the need for such a function, as well as provided ideas to how such a function could be implemented (Section~\ref{satisfactionResult}). It could be a button, which is a visible part of the board's GUI, placed away from the game map. This button could revert the last terraforming action taken in the game. An algorithm for this function could be a stacking system that remembers previous actions, allowing the program to revert them, returning to previous states of the program. Another solution could be an expansion of the terraforming action. Anyone, regardless of whose turn it is, should then be able to change a tile to any type of terrain in the game through a multiple choice menu. This would help users undo mistakes by letting them change the tiles back to what the player originally intended or back to their original state. A possible problem with the multiple choice solution is its dependency on players' memory regarding what the terrain of the hexes originally were before the misplaced terraforming. 

\subsection{Improvement of taking turns}
Participants of the usability test had ideas how to improve the turn taking further, besides improve in its programming. They mentioned that instead of switching player by pressing a hand against the middle of the surface, one should be able to switch turns by interacting with an area specifically only for switching turns. This will require the diffuser to be more evenly illuminated, which could be done by adding more IR lamps, making the originally planned player selection areas properly illuminated and thus usable. Properly illuminating the diffuser would also make BLOB detection more precise in general, possibly allowing players to use their own hands more rather than paper templates.

\subsection{Turn specification}
Test participants also noted they would like to have a visible indication of whose turn it would be next, and an overview of the turn order at any given time. Which player has the next turn could for example be visible on the game's GUI by highlighting the player's personal area of the board, or by colouring the space not used on the board with the player's colour, in order to emphasise which player currently is in charge of the turn.
Regarding the turn overview, a solution to that could be an overview implemented as part of the board's GUI. Its placement could either be as a part of the board's global information, just like the game objectives, in which case it would have to be large enough for each player to observe it clearly, or in front of every player, giving them each a equal personal overview of the turn order. In addition, one could design the overview so each overview graphic had focus on the player's placement in the turn order. The latter solution's advantage is an easier overview for all players, since none would have to look at the turn order from an odd angle. However, the first solution's advantage is that it might take up less space than having an overview for each player.

\subsection{Digital race/faction card}\label{sec:DigiRaceFact}
As one test participant mentioned, a fully digital version of the board game is possible. One could implement a graphical representation of the faction card and the information that normally follows in the original version of the game. As the faction card is made digital, it can also undergo changes in the way it visualises rules and upgrade costs. This could be implemented by making the icons for the tasks interactable, making the user able to show and hide tool tips explaining the task. If that text would take up too much space, making the cards too large on the board, a design idea could be to make up a tool tip box, specifically for text, which would change depending on which part of the race card graphics were interacted with. Another thing a digital faction card can implement is the possibility to upgrade a player's terraforming or shipping tasks, making them more effective. In the original game, this is done by moving a physical marker on the faction board. The digital faction board could be designed so it would change an upgradeable part when interacted with, for example by pressing the relevant part of the card, which would upgrade it and change its graphic representation.

\subsection{Management of resources}
Several test participants disliked the physical management of resources (power, gold, workers and priests), which can be remedied by managing the resources digitally. This could have the added benefit of increasing the game pace. This digital management of resources could work with the digital faction board mentioned in Section~\ref{sec:DigiRaceFact}, with tool-tips explaining the resources in detail. The resources could either be interactive, so the users change their amount themselves, or it could happen automatically as users choose to upgrade, build, terraform, etc. If the latter solution is to be implemented, the following can be considered:
\begin{itemize}
\item \textbf{Income:} Depending on the player's buildings and their racial powers and upgrades, a specific amount of gold, workers, priests and power should be received after the beginning of each round. If the augmented board could register the buildings each player owns, it could use this information to give each player their resource income every round. 
\item \textbf{Power reminder, offer and receiver:} This implementation requires for the augmented board to recognise the placement of buildings, and to keep trace of which players the buildings belongs to. Every time a player places a building next to one belonging to their opponent, the program should remind the players that power must be offered, and specify how much power is supposed to be offered, depending on the value and placement of the buildings. This graphic reminder could for example either be shown in its own area of the interface, or be shown as a pop-up, which could reinforce its ability to remind the players of what they need to do.
\item \textbf{Cult tracking:} As the cult tracker originally is done on its own board, it normally is not a part of the main game board. This could be changed in the augmented version, as one could design it to be a part of the interface. This information could be shown graphically either by having a tracker looking like the original one, or a more simplistic version, which would show the level of progress each player has in each cult. When specific milestones are reached in the cult tracker, the board could notify the players of what they gain from the milestones. This tracker could also, like in the original version, show the priests sacrificed and take care of the progression on tracker automatically every time a player did anything in order to progress.
\end{itemize}