%!TEX root = ../master.tex
\chapter{Initial Problem}\label{ch:iniprob}
In this section, the basic criteria for this semester project are explained. From this, the initial problem is defined.

The goal of this project is to develop an 'augmented' board game that makes use of computer vision. Based on this, as well as the semester requirements, five basic criteria for the project are defined:

\begin{enumerate}
	\item The product of this project must be a media artefact.
	\item The artefact must implement computer vision in such a way that makes it interactive.
	\item The artefact must include an existing board game that we have access to.
	\item The implemented computer vision must not make the act of playing more convoluted.
\end{enumerate}

The board game chosen for this project is Terra Mystica.

\section{A short introduction to Terra Mystica}
Terra Mystica is a strategy board game for 2-5 players about territory control and resource management. Each player can choose one of 14 races, each of which has their own playstyle. The board consists of hexagonal tiles  with different kinds of terrain. Players can only build structures on tiles with terrain which matches their race. This terrain can, however, be modified by players if they choose to spend the required resources in a process which is referred to as 'terraforming'. Finally, the game involves a power mechanic; if a player modifies a piece of terrain adjacent to the structures of another player, either through terraforming or through placing/upgrading their own buildings, that player can choose to gain power at the cost of victory points, which can be spent on acquiring resources.  

Victory points are gained based on objectives, some of which are global (obtainable by each player), while the others are race-specific. The winner is the player that has the most victory points at the end of the game.

The game involves many loose pieces for the players to manage (namely when terraforming), and the power mechanic is often forgotten by the players. \todo{This sentence seems disjointed from the rest, and out of place.}

\section{Initial Problem statement}
From this, an initial problem statement is defined: 

\textit{How can a media artefact that 'augments' Terra Mystica through an implementation of Computer Vision be developed?}

To answer this question, research questions are defined:

\textit{How can Computer Vision be used to facilitate user interaction?}

\textit{How can the software for a virtual board be developed?}

