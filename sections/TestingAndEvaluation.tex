%!TEX root = ../master.tex
\chapter{Testing and Evaluation}\label{ch:testeval}
writing something here to see if it works

\section{Purpose of test}
text here

\section{Test objectives}
text

\section{Method}
text\todo{will have many subsections probably}

\section{Results}
text\todo{analysis and conclusion}