%!TEX root = ../master.tex
\chapter{Final Problem}\label{ch:finprob}
\todo{some introduction here is needed}
\section{Formulation}
After having chosen Terra Mystica as the board game to augment, the augmentation was decided as being an interactive game board via computer vision.

The initial research questions were:
\begin{itemize}
	\item How can computer vision be used to facilitate user interaction 
	\item How should the software for that be developed.
\end{itemize}

As seen in previous scientific work with examples of current state of the art, Andersen et al.[ref] used recognition technology via camera and Peitz et al.[ref] showed the output from a game unto a projection. Both can add inspiration to how the initial research questions may be answered. An example being, creating an interactive projection, and having it being manipulated through interaction with its computer vision software.

The results of the ensuing background research were that BLOB recognition and background imaging in an RID solution would be a possible facilitation.
The software for that should be designed in such a way that the user can:
\begin{itemize}
\item Control the turn order
\item Start and end their own turn
\item Change a tile on the screen by 'terraforming'
\end{itemize}

In this project, an unutilised potential was identified when researching how augmentation like the current project idea would affect the usability of the game, in comparison to the non-augmented version. All that culminates into the final problem formulation:

\textit{“Can Terra Mystica become more usable through the use of a computer vision augmented board?”}

\subsection{Success Criteria}
\todo{this description seems vague, I might fix it later, else I am open for ideas}\todo{I did something new - Liv}
When defining the problem for the project, certain criteria were defined that needs to be met if the project is to be deemed a success. The success criteria have been defined as following:

\begin{itemize}
	\item The projection software must be able to make the image of the board game show one of seven different tiles in each interactive tile, and make tiles changeable, in order to make the user able to do terraforming in the game.
	\item The augmented game board must be usable for a whole game, so that the augmented version of Terra Mystica could replace the unaugmented one. That also means a game of Terra Mystica should not be halted or stopped by any technical problems.
	\item The computer vision software needs to register some sort of BLOB, in order to make the board game change the tiles. This means an implementation image processing that works with the projection of the board.
\end{itemize} \todo{maybe more can be added, but I don't know what}