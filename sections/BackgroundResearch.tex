%!TEX root = ../master.tex
\chapter{Background Research}\label{ch:bgres}

\section{Previous scientific work}
Andersen et al. \citep{andersen_designing_2004} created a board game utilising physical and digital components. For their physical components they used flat squares with patterns that through Augmented Reality (AR) technologies will be recognised through a camera and/or Virtual Reality goggles. Their research was made to illustrate "design issues for AR board games".

Peitz et al. \citep{peitzWizards2006} created "Wizard's Apprentice", a computer-augmented board game. The game includes two roles: wizard and apprentice. All players but one play as apprentices guided by the wizard who acts as a "negotiator and motivator. The software is written in Java and has several play modes. The game uses Radio-Frequency IDentification (RFID) hardware to detect at which physical point the players are in the game. It sends this information through antennas to a laptop close by. The laptop projects this information to a screen that displays relevant information on each player's turn.

\section{Methods for evaluating and measuring key success criteria}
Andersen et al. \citep{andersen_designing_2004} was evaluated on a group of 13 year old children. The goals were to "reveal issues concerning the game design in general, design of the physical pieces, the use of goggles versus screen display, and future evolvement." The children found the 3D projection of the characters through the goggles fascinating, although fumbling a bit with the game pieces occured due to the nature of the goggles and placement of the webcam. This occurred less when they looked at the 3D figures on the screen, but this brought along the problem of shifting focus away from the board game.
Finally, they noticed the children found the small set of animations uninteresting after a small while and suggested a higher number of animations were needed to keep interest.

Peitz et al. \citep{peitzWizards2006} performed an initial evaluation on 3 children aged 8-9 and a male adult player (37). The evaluation was based upon a template that evaluates the social adaptability of game divided into: "(...)spatial, temporal, social, and playability." The participants were observed for two hours and were thematically interviewed afterwards. The participants reported overall enjoyment and increased togetherness in the game, but the pacing of the game felt unfamiliar. The participants seemed to particularly react positively to the sound design. 


\section{Competitive analysis}
\todo {Pros and cons of aforementioned solutions}