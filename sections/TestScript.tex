%!TEX root = ../master.tex
\chapter{Evaluation script}\label{ch:TestScript}
\section{Introduction to test}
(To the whole group)
Hello, and thank you all for agreeing to participate in our test. We have created a touch board table on which you can play the board game Terra Mystica. In other words, our product is a digitally augmented version of Terra Mystica.

You will start out by playing three rounds of the game on the [digital/analogue] version. While you play, you are encouraged to talk amongst each other about the game, and voice your thoughts. Once the three rounds are up, we will switch to the [digital/analogue] version, on which you will also play three rounds.

When you have played the two games, we will hold a short interview where you will be asked questions about your thoughts towards the augmented game. In this interview, you will also be asked to compare this version to the original game.

\section{Introduction to Terra Mystica rules}
For the purpose of this test, you will play a simplified version of the game.

In Terra Mystica, you need three different resources: workers, gold, and priests. The goal of the game is to aquire the most victory points by the end.

If we start looking on your faction board, you have five types of buildings. Three of them give you resources at the start of a round:
\begin{itemize}
\item Houses, which give you workers
\item Trade posts, which give you gold
\item Temples, which give you priests
\end{itemize}
You also have two other buildings, which normally have special abilities, depending on your race. In this simplified game however, we use the following rules:
\begin{itemize}
\item The Stronghold, which gives you 7 victory points.
\item The Sanctuary, which gives you priests like temples, but also gives you 4 victory points.
\end{itemize}
The cost of each building is seen to the left of them on the faction board. The white squares are workers, and the coins are of course, coins. The little man-like shapes are priests.

In this simplified game, each Trade Post gives you 2 coins. Stronghold and Sanctuary costs 4 workers and 6 coins.

When it is your turn, you can do one of four things:
\begin{itemize}
\item You can build a House on tiles which are your own colour, and next to other buildings you own. 
\item You can transform a tile next to your buildings to your own colour. This action is called \textit{terraforming}. To terraform a tile of another colour, you look at the terraform circle on your faction board. Each jump from the colour you want to transform, to your own colour costs three workers.
\begin{itemize}
\item After you have terraformed a tile, you are allowed to build a house immediately, by paying its cost.
\end{itemize}
\item If you want to be able to terraform tiles for fewer workers, you can upgrade your terraform power. This costs coins, workers and a priest.
\item You can also upgrade a building, to get a more expensive one. Houses become Trade Posts, Trade Posts become either Temples or Strongholds, and Temples become Sanctuaries.
\end{itemize}

When you place a building, you always take buildings from left to right. 
When you upgrade, the old building is placed back on the faction board, going from right to left.

When you build a Trade Post next to another player, it costs three coins. When you only build Trade Posts next to your own buildings, it costs six coins.

The game is split into rounds and turns. In one round, each player makes one action in turn order. When the last player in the turn order takes an action, we go back to the starting player, and repeat. If you find that you can’t do anything in a turn, you can always say pass. When you pass, you drop out of the turn order, but start earlier in the next round. When all players have passed, the next round starts.

You get Victory Points by collecting the most coins, and having the largest cities.
In the first round, you get 2 Victory Points for each House you build.
In the second round, you get three Victory Points for each Trade Post you build.
In the third round, you get 5 Victory Points for each Sanctuary or Stronghold you build.

Any questions before we start?

\section{Explanation of the analogue version}
[Assuming they have played the digital version first]
The analogue version is played the same as the digital version, except you do not do a gesture when taking your turn, and you terraform by placing a token of your own colour on the tile you wish to terraform, like this [show it]. The buildings you wish to place are place on top of this token, like this [show it].

\section{Explanation of the digital version}
[Assuming they have played the analogue version first]
The digital version is played the same as the analogue version, except you now need to register yourself every time it is your turn, like this [show it], and instead of placing the tile tokens, you instead terraform like this[show it]. You will still place the buildings like before [show it].

\section{Post-test interview}
The following are points which needs coverage. This means that they should be used for a semistructured interview (meaning that the interviewer can, at any time, ask the interviewee to elaborate), made up by the questions raised from the testing and from the following: 
\begin{itemize}
\item How did you think the augmented Terra Mystica compares to the original game?
\begin{itemize}
\item Do you still feel like you were playing the same game when you were playing the augmented version?
\end{itemize}
\item How did you experience the act of indicating that is was your turn? Did you experience it as easy or difficult?
\begin{itemize}
\item How is the experience of taking turns in comparison to when playing the original game?
\end{itemize}
\item Did you experience ease or difficulty when terraforming compared to the original game?
\item Do you think the augmented game can serve as a possible replacement for the original board?
\item How do you think the game's pace was affected by the augmentation? Do you think it was increased or decreased?
\end{itemize}


