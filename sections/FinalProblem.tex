%!TEX root = ../master.tex
\chapter{Final Problem}\label{ch:finprob}
The initial problem formulation provided research questions to expand our understanding of the problem area. The following final problem formulation is instead focused on the development of our product.

\section{Problem Statement}
With Terra Mystica as the board game to augment, the augmentation will be an interactive game board via computer vision.

The initial research questions were:
\begin{itemize}
	\item How can computer vision be used to facilitate user interaction?
	\item How should the software for that be developed?
\end{itemize}

As seen in previous scientific work with examples of current state of the art, Andersen et al. \citep{andersen_designing_2004} used recognition technology via camera and Peitz et al. \citep{peitzWizards2006} showed the output from a game unto a projection. Both can add inspiration to how the initial research questions may be answered. An example being, creating an interactive projection, and having it being manipulated through interaction with its computer vision software.

The results of the ensuing background research were that BLOB recognition and background imaging in an RID solution would be a possible facilitation.
The software for that should be designed in such a way that the user can:
\begin{itemize}
\item Control the turn order
\item Start and end their own turn
\item Change a tile on the screen by 'terraforming'
\end{itemize}

In this project, an unutilised potential was identified when researching how augmentation like the current project idea would affect the usability of the game, in comparison to the non-augmented version. All that culminates into the final problem formulation:

\textit{“Can Terra Mystica become more usable through the use of a computer vision augmented board?”}

\subsection{Success Criteria}
\todo{this description seems vague, I might fix it later, else I am open for ideas}\todo{I did something new - Liv}
When defining the problem for the project, certain criteria were defined that need to be met if the project is to be deemed a success. The success criteria have been defined as following:

\textit{Usability:}
\begin{itemize}
	\item The project must have gone through evaluation and iterations, based on relevant usability principles.
\end{itemize}
\textit{Augmented Board:}
\begin{itemize}
	\item Augmentation technology must be utilised with Terra Mystica in such a way that it does not take away from the experience of the game but enhances it.
\end{itemize}
\textit{Computer Vision:}
\begin{itemize}
	\item Implemented augmentation technologies should utilise computer vision to augment Terra Mystica.
	\item The accuracy and effectiveness of the image detection should be tested in a proper framework.
\end{itemize} 