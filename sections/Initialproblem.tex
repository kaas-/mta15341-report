%!TEX root = ../master.tex
\chapter{Initial Problem}\label{ch:iniprob}
In this section, the basic criteria for this semester project will be explained. From this, the initial problem will be defined.

The goal of this project is to develop an 'augmented' board game that makes use of computer vision. Based on this, as well as the semester requirements, five basic criteria for the project are defined:

\begin{enumerate}
	\item The product of this project must be a media artefact.
	\item The artefact must implement computer vision in such a way that makes it interactive.
	\item The artefact must include an existing board game that we have access to.
	\item The chosen board game must not have an average game time of over two hours.
	\item The implemented computer vision must not make the act of playing more convoluted.
\end{enumerate}

The board game chosen for this project is Terra Mystica.

\section{A short introduction to Terra Mystica}
Terra Mystica is a strategy board game for 2-5 players about territory control and resource management. Each player can choose one of 14 races, each of which has their own playstyle. The board consists of hexes with different kinds of terrain. Players can only build structures on hexes with terrain which matches their race. This terrain can, however, be modified by players if they choose to spend the required resources. Finally, the game involves a Power mechanics; if a player modifies a piece of terrain next to the structures of another player, that player can choose to gain Power at the cost of Victory Points, which can be spent on acquiring resources.  

Victory Points are gained based on objectives, some of which are global (obtainable by each player), while the others are race-specific. The winner is the player that has the most Victory Points at the end of the game.

\section{An interactive board for Terra Mystica}
This project aims to create an interactive board for Terra Mystica, that uses Computer Vision to detect hand contact on the board via a camera below the surface of it, and the shapes and colours of specific game pieces via a camera from above. The artefact will then use this information to manipulate the projected game board. This can be for handling terrain changes, and for projecting information such as the amount of Power one will receive if a player chooses to place game pieces next to those of another player that are already placed. Furthermore, it can handle the calculation of points gained from global objectives at the end of the game.

\section{Problem statement}
From this, an initial problem statement is defined: 

\textit{How can a media artefact that 'augments' Terra Mystica through an implementation of Computer Vision be developed?}

To answer this question, research questions are defined:

\textit{How can Computer Vision be used to facilitate user interaction?}

\textit{How can the software for a virtual board be developed?}

\subsection{Target Group Explained}

\section{Final Problem Formulation}\label{sec:finalprob}

